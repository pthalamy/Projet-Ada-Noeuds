\documentclass[12pt]{article}

\usepackage[french]{babel}
\usepackage[utf8]{inputenc}
\usepackage{fancyhdr}
\usepackage{lastpage}
\usepackage{graphicx}
\usepackage[T1]{fontenc}
\usepackage{amsmath,amssymb}
\usepackage{fullpage}
\usepackage{url}
\usepackage{xspace}

\usepackage{listings}
\lstset{
  morekeywords={abort,abs,accept,access,all,and,array,at,begin,body,
    case,constant,declare,delay,delta,digits,do,else,elsif,end,entry,
    exception,exit,for,function,generic,goto,if,in,is,limited,loop,
    mod,new,not,null,of,or,others,out,package,pragma,private,
    procedure,raise,range,record,rem,renames,return,reverse,select,
    separate,subtype,task,terminate,then,type,use,when,while,with,
    xor,abstract,aliased,protected,requeue,tagged,until},
  sensitive=f,
  morecomment=[l]--,
  morestring=[d]",
  showstringspaces=false,
  basicstyle=\small\ttfamily,
  keywordstyle=\bf\small,
  commentstyle=\itshape,
  stringstyle=\sf,
  extendedchars=true,
  columns=[c]fixed
}

% CI-DESSOUS: conversion des caractères accentués UTF-8 
% en caractères TeX dans les listings...
\lstset{
  literate=%
  {À}{{\`A}}1 {Â}{{\^A}}1 {Ç}{{\c{C}}}1%
  {à}{{\`a}}1 {â}{{\^a}}1 {ç}{{\c{c}}}1%
  {É}{{\'E}}1 {È}{{\`E}}1 {Ê}{{\^E}}1 {Ë}{{\"E}}1% 
  {é}{{\'e}}1 {è}{{\`e}}1 {ê}{{\^e}}1 {ë}{{\"e}}1%
  {Ï}{{\"I}}1 {Î}{{\^I}}1 {Ô}{{\^O}}1%
  {ï}{{\"i}}1 {î}{{\^i}}1 {ô}{{\^o}}1%
  {Ù}{{\`U}}1 {Û}{{\^U}}1 {Ü}{{\"U}}1%
  {ù}{{\`u}}1 {û}{{\^u}}1 {ü}{{\"u}}1%
}

%%%%%%%%%%
% TAILLE DES PAGES (A4 serré)

\setlength{\parindent}{0pt}
\setlength{\parskip}{1ex}
\setlength{\textwidth}{17cm}
\setlength{\textheight}{24cm}
\setlength{\oddsidemargin}{-.7cm}
\setlength{\evensidemargin}{-.7cm}
\setlength{\topmargin}{-.5in}

%%%%%%%%%%
% EN-TÊTES ET PIED DE PAGES

%% \pagestyle{fancyplain}
\renewcommand{\headrulewidth}{0pt}
\addtolength{\headheight}{1.6pt}
\addtolength{\headheight}{2.6pt}
\lfoot{}
\cfoot{\footnotesize\sf TPL Algo - Nœuds}
\rfoot{\footnotesize\sf page~\thepage/\pageref{LastPage}}
\lhead{}


%%%%%%%%%%
% COMMANDES PERSONNALISEES
\newcommand{\shellcmd}[1]{\\\indent\indent\texttt{\footnotesize\# #1}\\}

%%%%%%%%%%
% TITRE DU DOCUMENT

\title{Rapport de Projet Algo -- Nœud}
\author{\textsc{Benjamin Lebit} - \textsc{Pierre Thalamy}}
\date{\today}

\begin{document}

\maketitle

\section {Vue d'ensemble du projet}
Tous d'abord, nous avons bien implémenté dans notre programme tout ce
que décrivait l'énoncé. Cependant, nous avons en plus décidé de supporter à la
fois le tracé des nœuds selon l'angle \b{max} et \b{min}. C'est pourquoi, afin
d'exécuter le programme, il faudra utiliser la commande :
\shellcmd{./noeuds banana\_X.kn output.svg min} 
pour tracer selon l'angle minimum, ou, pour l'angle maximum :
\shellcmd{./noeuds banana\_X.kn output.svg max} 

Le projet comprend 6 modules :
\begin{enumerate}
\item \emph{Noeuds} : Le main faisant appels aux procédures de tous
  les autres modules.
\item \emph{Objets} : Comprend les définitions de tous les objets et
  structures de données utilisées dans le projet, ainsi que des
  procédures pour en afficher le contenu.
\item \emph{Liste} : Le module de gestion de liste, plus de détails en
  section 2.
\item \emph{Parseur} : Procédures d'extraction des données des fichiers
  d'entrée pour stockage dans les structures appropriées.
\item \emph{Traitement} : Contient les procédures d'exploitation des
  données stockées (Calculs des points de contrôle, milieu
  suivant...).
\item \emph{Svg} : Réalise le tracé intermédiaire et coordonne le tracé des nœuds.
\end{enumerate}

\section {Modélisation des données}
La structure de donnée principale du programme est \emph{Tab\_Sommets},
un tableau indexé de 1 au nombre de sommet du programme et qui stocke
des elements de types \emph{Sommet}. Un sommet est un \emph{Point}, de
coordonnée \emph{X, Y}, et une liste chaîné de voisins. \\
Cette liste, nommé \emph{Liste\_Voisin}, contient pour chaque cellule,
un element de type \emph{Arrete}, ainsi que l'indice du voisin qui lui
est associée, afin de la repérer aisément. \\
Enfin, une \emph{Arrete} contient l'ID du sommet qui la stocke et celui
qui lui est opposé, ainsi que les points de contrôle de chacun sur
cette arête. On stocke aussi le milieu et la longueur du sommet, pour
éviter d'avoir à les recalculer à chaque utilisation. L'inconvénient
principal de ce choix de structure est que le stockage des arêtes est
redondant. Le sommet 1 aura dans sa liste l'arête 1--2, et 2 aura
2--1 dans sa liste.

\section {Tracé intermédiaire}
L'ensemble du tracé est assuré par la fonction Sauvegarde du paquet svg.
\\
Le tracé intermédiaire du graphe (arêtes et croix sur leurs milieux) est effectué après le parsing du fichier kn d'entré par la fonction
\emph{Trace\_Intermediaire} du paquet svg, à partir du tableau de
sommets rempli précédemment, et des points de contrôle générés par la
procédure \emph{Stocker\_Points\_De\_Controle} du module
\emph{Traitement}. 
Nous avons choisi de tracer les croix en bleu pour une meilleur
distinction avec les arêtes sur le rendu svg.\\
\\
Afin d'éviter de tracer deux fois les arêtes à cause de la redondance
évoquée dans la section précédente, on utilise un tableau de marquage à deux
dimensions, de même indexation que le tableau de sommets. Avant de
tracer l'arête I--J, on vérifie que T(I, J) et T(J, I) vaut \emph{false},
sinon on ignore.
\section {Tracé des nœuds}
Le tracé des nœuds est effectué par la procedure \emph{Trace\_Noeuds} du
paquet \emph{Svg}.On commence le tracé à partir de la première arête du premier sommet du tableau de sommets. Cette fonction appelle d'autres procédures de calcul du paquet traitement pour calculer et/ou récupérer les coordonnées de 4 points :
\begin{itemize}
\item Ceux du point de départ, le milieu de l'arête courante du sommet considéré.
\item Ceux du point d'arrivé, récupéré par l'intermediaire de l'arrete cible, elle même déterminée par l'angle fait avec l'arête courante (le sens de celui-ci dépendera du sens de rotation, défini par le booléen Trig). L'arête cible sera choisie si l'angle qu'elle fait avec l'arête courante est minimum ou maximum (selon la valeur du booléen Min passée en argument lors de l'exécution).
\item Ceux du point de contrôle de départ qui dépenderont du sommet considéré sur l'arête courante ainsi que du sens dans lequel sera calculé l'angle (valeur du booléen Trig).
\item Ceux du point de contrôle d'arrivée qui dépenderont du sommet considéré sur l'arête cible ainsi que du sens de rotation.
\end{itemize}
Une fois calculés, ces coordonnées seront utilisées dans la fonction
\emph{Trace\_Bezier} pour le tracé des nœuds dans le fichier de sortie. 
Afin de déterminé la fin du tracé, nous incrémentons à chaque tracé un
compteur allant de 0 à 2$\times$(Nombre\verb+_+Sommets-1). 
\end{document}
