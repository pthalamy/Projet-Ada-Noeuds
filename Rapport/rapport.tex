\documentclass[10.9pt]{article}

\usepackage[french]{babel}
\usepackage[utf8]{inputenc}
\usepackage{fancyhdr}
\usepackage{lastpage}
\usepackage{graphicx}
\usepackage[T1]{fontenc}
\usepackage{amsmath,amssymb}
\usepackage{fullpage}
\usepackage{url}
\usepackage{xspace}

\usepackage{listings}
\lstset{
  morekeywords={abort,abs,accept,access,all,and,array,at,begin,body,
    case,constant,declare,delay,delta,digits,do,else,elsif,end,entry,
    exception,exit,for,function,generic,goto,if,in,is,limited,loop,
    mod,new,not,null,of,or,others,out,package,pragma,private,
    procedure,raise,range,record,rem,renames,return,reverse,select,
    separate,subtype,task,terminate,then,type,use,when,while,with,
    xor,abstract,aliased,protected,requeue,tagged,until},
  sensitive=f,
  morecomment=[l]--,
  morestring=[d]",
  showstringspaces=false,
  basicstyle=\small\ttfamily,
  keywordstyle=\bf\small,
  commentstyle=\itshape,
  stringstyle=\sf,
  extendedchars=true,
  columns=[c]fixed
}

% CI-DESSOUS: conversion des caractères accentués UTF-8 
% en caractères TeX dans les listings...
\lstset{
  literate=%
  {À}{{\`A}}1 {Â}{{\^A}}1 {Ç}{{\c{C}}}1%
  {à}{{\`a}}1 {â}{{\^a}}1 {ç}{{\c{c}}}1%
  {É}{{\'E}}1 {È}{{\`E}}1 {Ê}{{\^E}}1 {Ë}{{\"E}}1% 
  {é}{{\'e}}1 {è}{{\`e}}1 {ê}{{\^e}}1 {ë}{{\"e}}1%
  {Ï}{{\"I}}1 {Î}{{\^I}}1 {Ô}{{\^O}}1%
  {ï}{{\"i}}1 {î}{{\^i}}1 {ô}{{\^o}}1%
  {Ù}{{\`U}}1 {Û}{{\^U}}1 {Ü}{{\"U}}1%
  {ù}{{\`u}}1 {û}{{\^u}}1 {ü}{{\"u}}1%
}

%%%%%%%%%%
% TAILLE DES PAGES (A4 serré)

\setlength{\parindent}{0pt}
\setlength{\parskip}{1ex}
\setlength{\textwidth}{17cm}
\setlength{\textheight}{24cm}
\setlength{\oddsidemargin}{-.7cm}
\setlength{\evensidemargin}{-.7cm}
\setlength{\topmargin}{-.5in}

%%%%%%%%%%
% EN-TÊTES ET PIED DE PAGES

%% \pagestyle{fancyplain}
\renewcommand{\headrulewidth}{0pt}
\addtolength{\headheight}{1.6pt}
\addtolength{\headheight}{2.6pt}
\lfoot{}
\cfoot{\footnotesize\sf TPL Algo - Nœuds}
\rfoot{\footnotesize\sf page~\thepage/\pageref{LastPage}}
\lhead{}


%%%%%%%%%%
% COMMANDES PERSONNALISEES
\newcommand{\shellcmd}[1]{\\\indent\indent\texttt{\footnotesize\# #1}\\}

%%%%%%%%%%
% TITRE DU DOCUMENT

\title{Rapport de Projet Algo -- Nœud}
\author{\textsc{Benjamin Lebit} - \textsc{Pierre Thalamy}}
\date{\today}

\begin{document}

\maketitle

\section {Vue d'ensemble du projet}
Tous d'abord, nous avons bien implémenté dans notre programme tout ce
que décrivait l'énoncé. Cependant, nous avons décidé de supporter à la
fois le nouage selon l'angle \b{max} et \b{min}. C'est pourquoi, afin
d'exécuter le programme, il faudra utiliser la commande :
\shellcmd{./noeuds banana\_X.kn output.svg min} 
pour tracer selon l'angle minimum, ou, pour l'angle maximum :
\shellcmd{./noeuds banana\_X.kn output.svg max} 

Le projet comprend 6 modules :
\begin{enumerate}
\item \em{Noeuds} : Le main faisant appels aux procédures de tous
  les autres modules.
\item \em{Objets} : Comprend les définitions de tous les objets et
  structures de données utilisées dans le projet, ainsi que des
  procédures pour en afficher le contenu.
\item \em{Liste} : Le module de gestion de liste, plus de détails en
  section 2.
\item \em{Parseur} : Procédures d'extraction des données des fichiers
  d'entrée pour stockage dans les structures appropriées.
\item \em{Traitement} : Contient les procédures d'exploitation des
  données stockées (Calculs des points de contrôle, milieu
  suivant...).
\item \em{Svg} : Réalise le tracé intermédiaire et coordonne le tracé des nœuds.
\end{enumerate}

\section {Modélisation des données}
La structure de donnée principale du programme est \em{Tab\_Sommets},
un tableau indexé de 1 au nombre de sommet du programme et qui stocke
des elements de types \em{Sommet}. Un sommet est un \em{Point}, de
coordonnée \em{X, Y}, et une liste chaîné de voisins. \\
Cette liste, nommé \em{Liste\_Voisin}, contient pour chaque cellule,
un element de type \em{Arrete}, ainsi que l'indice du voisin qui lui
est associée, afin de la repérer aisément. \\
Enfin, une \em{Arrete} contient l'ID du sommet qui la stocke et celui
qui lui est opposé, ainsi que les points de contrôle de chacun sur
cette arrète. On stocke aussi le milieu et la longueur du sommet, pour
éviter d'avoir à les recalculer à chaque utilisation. L'inconvénient
principal de ce choix de structure est que le stockage des arrètes est
redondant. Le sommet 1 aura dans sa liste l'arrête 1--2, et 2 aura
2--1 dans sa liste.

\section {Tracé intermédiaire}

\section {Tracé des nœuds}

\end{document}
